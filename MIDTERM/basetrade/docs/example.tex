\documentclass[10pt,a4paper]{article}
\usepackage[utf8]{inputenc}
\usepackage{amsmath}
\usepackage{amssymb}
\title{Walk-forward Approach to High Frequency Trading}

\begin{document}
	
\maketitle

\begin{abstract}
	We show instances where a portfolio allocated according to Modern Portfolio Theory (MPT), i.e. Mean-Variance-Optimization (MVO), achieves significantly lower returns than other baseline strategies like constant-rebalanced portfolios. We talk about the main drawbacks of MVO and how subsequent quantitative approaches have improved upon them. In this paper, we have tried to correct the false messaging that has been proliferated to investors around the optimality of MVO in real-world investing conditions. At the end, we take the beauty of MPT and the rigor of data-science to make a successful, thorough, and yet cost-effective investment management system.
\end{abstract}




\section{Introduction and Motivation}

Traders have observed that re-tuning their strategies to recent data often is useful in obtaining better strategies. One obvious case for a trader in high frequency trading is that his\footnote{In this document we assume that the trader is male} product standard-deviation has changed recently and one would like to adapt in a dynamic fashion to this change. The other case would be when your product has started following a different source. A possible ill-effect of this is that strategies are overfitted to more recent data and there is no way to prevent this as traders refine their strategies. Recent parameter setting for entry-exit of trades also do better than older parameter settings. All of these observation lead us to think about a walkforward setup where we think of configs as settings






\section{Ways to think about Walk-forward}
\end{document}